%----------------------------------------------------------------------------------------
%	PACKAGES AND OTHER DOCUMENT CONFIGURATIONS
%----------------------------------------------------------------------------------------

\documentclass[a0,portrait]{a0poster}

\usepackage{multicol} % This is so we can have multiple columns of text side-by-side
\columnsep=100pt % This is the amount of white space between the columns in the poster
\columnseprule=3pt % This is the thickness of the black line between the columns in the poster

\usepackage[svgnames]{xcolor} % Specify colors by their 'svgnames', for a full list of all colors available see here: http://www.latextemplates.com/svgnames-colors

\usepackage{times} % Use the times font
%\usepackage{palatino} % Uncomment to use the Palatino font

\usepackage{graphicx} % Required for including images
\graphicspath{{figures/}} % Location of the graphics files
\usepackage{booktabs} % Top and bottom rules for table
\usepackage[font=small,labelfont=bf]{caption} % Required for specifying captions to tables and figures
\usepackage{amsfonts, amsmath, amsthm, amssymb} % For math fonts, symbols and environments
\usepackage{wrapfig} % Allows wrapping text around tables and figures
\usepackage[export]{adjustbox}

\begin{document}

%----------------------------------------------------------------------------------------
%	POSTER HEADER 
%----------------------------------------------------------------------------------------

% The header is divided into two boxes:
% The first is 75% wide and houses the title, subtitle, names, university/organization and contact information
% The second is 25% wide and houses a logo for your university/organization or a photo of you
% The widths of these boxes can be easily edited to accommodate your content as you see fit

\begin{minipage}[t]{0.75\linewidth}
\VeryHuge \color{HoGentAccent1} \textbf{The migration process and advantages of Windows Server 2019} \color{Black}\\ % Title
%\Huge\textit{Ondertitel (eventueel)}\\[2.4cm] % Subtitle
\huge \textbf{Du Four Jens, Vershuere Glenn, Stroobants Ludwig}\\[0.5cm] % Author(s)
\huge Hogeschool Gent, Valentin Vaerwyckweg 1, 9000 Gent\\[0.4cm] % University/organization
\Large \texttt{jens.dufour.y2321@student.hogent.be} \\
\end{minipage}
%
\begin{minipage}[t]{0.25\linewidth}
\includegraphics[width=13cm,right]{figures/HOGENT_Logo_Pos_rgb.png} 

\end{minipage}

\vspace{1cm} % A bit of extra whitespace between the header and poster content

%----------------------------------------------------------------------------------------

\begin{multicols}{2} % This is how many columns your poster will be broken into, a portrait poster is generally split into 2 columns

%----------------------------------------------------------------------------------------
%	ABSTRACT
%----------------------------------------------------------------------------------------

\color{HoGentAccent1} % Navy color for the abstract

\begin{abstract}
This research covers the topic of Windows Server 2019, the different versions, and the advantages it has to its precursor Windows Server 2016.
Windows Server is one of the most used enterprise operating systems in the world. 
This makes keeping it up-to-date in the fast paced world of information technologies crucial. 
Thanks to the abundance of new features it offers looking into migration to this newer version is definitely an important question.
This research paper will look into how the migration process from Windows Server 2016 to Windows Server 2019 would work, the advantages and possible disadvantages of this  process. 
Another important part will be looking into the migration of an SAP environment, as described by delaware.
At last, there will be a comparison between the different available base container images and how these can be leveraged.
In the end this research paper will advise for or against a migration to the latest version of the operating system.
\end{abstract}
%----------------------------------------------------------------------------------------
%	INTRODUCTION
%----------------------------------------------------------------------------------------

\color{HoGentAccent1} 
\section*{Introductie}
\color{black}
\color{black}
Windows Server is a well-known operating system among IT professionals. With major organizations all around the globe, like Infosys, that have implemented some form of the operating system in their infrastructure.
One of the tasks that often need to be performed on the operating system, is keeping applications up-to-date. 
However, it is not often described how to migrate to a newer version.  
This does not mean that frequent updates are not important for both the addition of new features as well as the improvement of security, as this is becoming one of the big concerns of this century. 
This bachelor's thesis will go in-depth about the advantages and disadvantages of migrating to the latest version of Windows Server.
In specific, this bachelor's thesis look at the migration from Windows Server 2016 to Windows Server 2019, with as an optional requirement the migration of a SAP environment, as this is applicable to the business environment of the originator, delaware.
The key purpose is to find a procedure that is efficient, minimizes possible downtime and is applicable not only to the originator, delaware, but also to other organizations that find them self in the same situation and want to migrate their infrastructure to the latest version of the operating system.
%----------------------------------------------------------------------------------------
%	GEOLOGY
%----------------------------------------------------------------------------------------

\color{Black} % DarkSlateGray color for the rest of the content
\color{HoGentAccent1} 
\section*{Experimenten}
\color{black}
A newt? Camelot! Why? No, no, no! Yes, yes. A bit. But she's got a wart.

Shut up! I dunno. Must be a king. Who's that then? Look, my liege! On second thoughts, let's not go there. It is a silly place.

Shut up! Will you shut up?! No, no, no! Yes, yes. A bit. But she's got a wart. He hasn't got shit all over him. It's only a model. It's only a model.

Bring her forward! I don't want to talk to you no more, you empty-headed animal food trough water! I fart in your general direction! Your mother was a hamster and your father smelt of elderberries! Now leave 



\color{HoGentAccent1} 
\section*{Sectie met figuur}
\color{black}


\begin{center}\vspace{1cm}
\includegraphics[width=1.0\linewidth]{grail}
\captionof{figure}{\color{HoGentAccent5} He hasn't got shit all over him. The nose? Where'd you get the coconuts? What do you mean? We shall say 'Ni' again to you, if you do not appease us}
\end{center}\vspace{1cm}

%------------------------------------------------



\color{HoGentAccent1} 
\section*{Conclusies}
\color{black}
on't underestimate the Force. Oh God, my uncle. How am I ever gonna explain this? I suggest you try it again, Luke. This time, let go your conscious self and act on instinct. Don't be too proud of this technological terror you've constructed. The ability to destroy a planet is insignificant next to the power of the Force.
%----------------------------------------------------------------------------------------
%	FORTHCOMING RESEARCH
%----------------------------------------------------------------------------------------
\color{HoGentAccent1} 
\section*{Toekomstig onderzoek}
\color{black}

I care. So, what do you think of her, Han? No! Alderaan is peaceful. We have no weapons. You can't possibly… I have traced the Rebel spies to her. Now she is my only link to finding their secret base.

Kid, I've flown from one side of this galaxy to the other. I've seen a lot of strange stuff, but I've never seen anything to make me believe there's one all-powerful Force controlling everything. There's no mystical energy field that controls my destiny. It's all a lot of simple tricks and nonsense. You are a part of the Rebel Alliance and a traitor! Take her away! 


%----------------------------------------------------------------------------------------

\end{multicols}
\end{document}