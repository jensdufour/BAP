% !TeX spellcheck = en_GB
%%=============================================================================
%% Conclusie
%%=============================================================================
\chapter{\IfLanguageName{dutch}{Conclusie}{Conclusion}}
\label{ch:conclusie}
The conclusion of this bachelor's thesis is that a migration to Windows Server 2019 is achievable.
The latest version of the \acrshort{os} offers a smooth and manageable experience through the advancements made in the four key themes of Windows Server 2019. 
The extended support till 2029 makes it an ideal choice to future-proof any organization. 
While a migration to Windows Server 2019 in an environment with a third-party application requires additional research, it was attainable for a SAP environment. 
The migration from Windows Server 2016 to Windows Server 2019 is also feasible for any organization running an \acrshort{eol} \acrshort{os}, as expected with the latest version of the \acrshort{os}.
This for either the general \acrshort{os} as for the base container images.
Windows Server 2019 offers a wide array of improvements in terms of security, hybrid cloud, application platform and the \acrlong{hci}.
The addition of the new Windows base container image provides the tools for automated \acrshort{ui} tests and its additional dependencies make it perfect for utilization with out-of-date packages. 
The Server Core base container image can be used to run typical Windows services. 
It offers application compatibility and has a wide array of built-in Windows roles and features. 
The Nano Server base container image was designed for 'born in the cloud' applications that provide an agile deployment and offer on-demand availability. 
The reduced footprint of the base container images without a drop of performance make the latest version the natural choice. 
The new features which are introduced can be leveraged through the usage of the \acrfull{wac}.
Organizations that are running \acrshort{eol} \acrlong{os}s should consider the migration to Windows Server 2019.
Organizations that are running \acrshort{eol} infrastructure could also consider a migration to the cloud, although this imposes additional research towards the advantages of a cloud solution in comparison to an in-house infrastructure.
