% !TeX spellcheck = en_GB
%%=============================================================================
%% Inleiding
%%=============================================================================
\chapter{\IfLanguageName{dutch}{Inleiding}{Introduction}}
\label{ch:inleiding}
Windows Server is an \acrfull{os} that is widely used by organizations all over the world. Some of them with only a handful of employees to corporations that have a couple of thousand at their disposal, in multiple geographic locations. To keep up in the fast-paced world that is \acrfull{it}, updates are a necessary part of the daily operation, though they do not always present themselves at the most convenient time. 
\acrfull{sac} updates which can be scheduled to \acrfull{ltsc}, that could require entire systems to be taken offline for an extended duration. This can bring tremendous damage to the business value of the organization since \acrshort{it} has become a core business for many of them. Therefore, asking how this can be done as efficient and cost-effective as possible is important before considering a migration to a newer version. 

\section{\IfLanguageName{dutch}{Probleemstelling}{Problem Statement}}
\label{sec:probleemstelling}
Since the release of Windows Server 2019, the latest version, many organizations are willing to look into the migration to this version from previous versions in the nearby future. One of these organizations, Delaware, wants to research how this migration would take place, starting from Windows Server 2016, and how the migration can be achieved in an efficient and cost-effective way. 

\begin{table}[ht]
	\centering
	\begin{adjustbox}{width=1\textwidth}
		\begin{tabular}{l|l|l|ll}
			Products Released & Life cycle Start Date & Mainstream Support End Date & Extended Support End Date &\\
			\hline
			Windows Server 2016 Standard & 15/10/2016 & 11/01/2022 & 12/01/2027 &\\
			Windows Server 2016 Datacenter & 15/10/2016 & 11/01/2022 & 12/01/2027 &\\
		\end{tabular}
	\end{adjustbox}
	\caption[\acrshort{eol} \acrshort{ws}2016]{\acrshort{eol} of Windows Server 2016}
	\scriptsize	
	Adapted from \cite{MicrosoftEOL2019}
	\label{tab:EOL2016}
\end{table}

With the ending of mainstream support for Windows Server 2016 scheduled at the start of 2022, as seen in Table ~\ref{tab:EOL2016}, it is best to research the migration to the latest version well in advance. Especially when considering that organizations running \acrfull{eol} \acrshort{os}s are very common. In an article written by \textcite{Tsai2016}, it is shown that the market share, in the Spiceworks network, of Windows Server 2003 was at 17.9\% and that of Windows Server 2008 was at 45.4\%, in 2016. The official support for both of these versions has already ended. The extended support for the latter will be ending at the start of 2020. 

\begin{table}[ht]
	\centering
	\begin{adjustbox}{width=1\textwidth}
		\begin{tabular}{l|l|l|ll}
			Products Released & Life cycle Start Date & Mainstream Support End Date & Extended Support End Date &\\
			\hline
			Windows Server 2019 Essentials & 13/11/2018 & 09/01/2024 & 09/01/2029 &\\
			Windows Server 2019 Standard & 13/11/2018 & 09/01/2024 & 09/01/2029 &\\
			Windows Server 2019 Datacenter & 13/11/2018 & 09/01/2024 & 09/01/2029 &\\
		\end{tabular}
	\end{adjustbox}
	\caption[\acrshort{eol} \acrshort{ws}2019]{\acrshort{eol} of Windows Server 2019}
	\scriptsize	
	Adapted from \cite{MicrosoftEOL2019}
	\label{tab:EOL2019}
\end{table}

Those deadlines combined with the \acrshort{eol} of Windows Server 2016, are making a strong case to look into a migration to a newer version, which can offer enhanced security and additional features. The life cycle of the latest version, Windows Server 2019, started in November 2018 and it will continue to receive mainstream support until January 2024. The extended support is guaranteed until January 2029, as seen in Table ~\ref{tab:EOL2019}. This makes it a viable successor for the \acrshort{os} that makes up a large percent of the server infrastructure used in the world. This paper attempts to show that the benefits will outweigh the disadvantages that are paired with a migration. The advantages of the latest version can be divided into four key themes:

\begin{itemize}
	\item Hybrid cloud
	\item Security
	\item Application platform
	\item Hyper-converged infrastructure
\end{itemize}

Those four themes will be examined very thoroughly in the state of affairs.

\section{\IfLanguageName{dutch}{Onderzoeksvraag}{Research question}}
\label{sec:onderzoeksvraag}
What are the advantages and disadvantages of a migration from Windows Server 2016 to 2019 in a business environment?

\subsection{Sub-research question}

\begin{itemize}
	\item What are the differences between the standard, ServerCore and Nano version of Windows Server 2019?
	\item Can SAP be migrated from an existing Windows Server 2016 to 2019 in a business environment?
	\item How can the new features of Windows Server 2019 be leveraged in the migrated infrastructure? 
\end{itemize}

\section{\IfLanguageName{dutch}{Onderzoeksdoelstelling}{Research objective}}
\label{sec:onderzoeksdoelstelling}

The expected result of this bachelor's thesis is to demonstrate the advantages of utilizing a newer version of Windows Server. Proving that migration is a long-term investment towards the future. It will also compare the Server, ServerCore and Nano version of the \acrshort{os}. However, migrating an SAP environment will not yet be recommended due to the new nature of Windows Server 2019. This due to requirements that have not yet been met between the \acrshort{os} and the software solution. 

\section{\IfLanguageName{dutch}{Opzet van deze bachelorproef}{Structure of this bachelor thesis}}
\label{sec:opzet-bachelorproef}

The remainder of this bachelor's thesis is structured as follows:

In Chapter ~\ref{ch:stand-van-zaken}, an overview is given of the state of affairs within the research domain, based on a literature study.\\

In Chapter ~\ref{ch:methodologie}, a methodology is explained and the research techniques used to formulate an answer to the research questions are discussed.\\

%In Chapter ~\ref{ch:toekomstvisie}, the further development of Windows Server over the years to come will be analysed. \\

In Chapter~\ref{ch:conclusie}, finally, the conclusion is given and an answer is formulated to the research questions. This will also provide an impetus for future research within this domain.\\
