% !TeX spellcheck = en_GB
%%=============================================================================
%% Inleiding
%%=============================================================================
\chapter{\IfLanguageName{dutch}{Inleiding}{Introduction}}
\label{ch:inleiding}
Windows Server is an \acrfull{os} that is widely used by organizations all over the world, some with only a handful of employees to corporations that have a couple of thousand at their disposal. 
To keep up in the fast-paced world that is \acrfull{it}, updates are a necessary part of the daily operation, though they do not always present themselves at a convenient time. 
\acrfull{sac} updates which can be scheduled to \acrfull{ltsc} updates, that could require entire systems to be taken offline for an extended duration. 
Since \acrshort{it} has become a core business, this can bring tremendous damage to the business value of many of these organizations.
Therefore, asking how this can be done as efficient and cost-effective as possible is important before considering a migration to the latest version. 

\section{\IfLanguageName{dutch}{Probleemstelling}{Problem statement}}
\label{sec:probleemstelling}
Since the latest version of the Windows Server was released, Windows Server 2019, many organizations are willing to investigate a migration from previous versions of the \acrshort{os} in the nearby future. 
delaware, one of these organizations, wants to research how these migrations would take place starting from Windows Server 2016 and how they can be achieved in an efficient and cost-effective way. 

\begin{table}[htb!]
	\centering
	\begin{adjustbox}{width=1\textwidth}
		\begin{tabular}{l|l|l|ll}
			Products Released & Life Cycle Start Date & Mainstream Support End Date & Extended Support End Date &\\
			\hline
			Windows Server 2016 Standard & 15/10/2016 & 11/01/2022 & 12/01/2027 &\\
			Windows Server 2016 Datacenter & 15/10/2016 & 11/01/2022 & 12/01/2027 &\\
		\end{tabular}
	\end{adjustbox}
	\caption[\acrshort{eol} Windows Server 2016]{\acrshort{eol} of Windows Server 2016}
	\scriptsize	
	Adapted from \cite{MicrosoftEOL2019}
	\label{tab:EOL2016}
\end{table}

With the ending of mainstream support for Windows Server 2016 scheduled at the start of 2022, as seen in Table \ref{tab:EOL2016}, it is best to research the migration to the latest version well in advance. 
Especially when considering that organizations running an \acrfull{eol} \acrshort{os}, are very common. \\
In an article written by \textcite{Tsai2016}, the Spiceworks network showed that the market share of Windows Server 2003 was at 17.9\% and Windows Server 2008 was at 45.4\%. 
The official support for both versions has already ended. The extended support for the latter will be ending at the start of 2020. 

\begin{table}[ht]
	\centering
	\begin{adjustbox}{width=1\textwidth}
		\begin{tabular}{l|l|l|ll}
			Products Released & Life Cycle Start Date & Mainstream Support End Date & Extended Support End Date &\\
			\hline
			Windows Server 2019 Essentials & 13/11/2018 & 09/01/2024 & 09/01/2029 &\\
			Windows Server 2019 Standard & 13/11/2018 & 09/01/2024 & 09/01/2029 &\\
			Windows Server 2019 Datacenter & 13/11/2018 & 09/01/2024 & 09/01/2029 &\\
		\end{tabular}
	\end{adjustbox}
	\caption[\acrshort{eol} Windows Server 2019]{\acrshort{eol} of Windows Server 2019}
	\scriptsize	
	Adapted from \cite{MicrosoftEOL2019}
	\label{tab:EOL2019}
\end{table}

Those deadlines combined with the \acrshort{eol} of Windows Server 2016, make a strong case to investigate the migration to the latest version, which can offer enhanced security and additional features. 
The life cycle of the latest version, Windows Server 2019, started in November 2018 and it will continue to receive mainstream support until January 2024. 
The extended support is guaranteed until January 2029, as seen in Table ~\ref{tab:EOL2019}.\\
This makes it a viable successor for the \acrshort{os} that makes up a large percent of the server infrastructure used in the world. 
This bachelor's thesis attempts to show that the benefits that are paired with migrating will outweigh the disadvantages. 
The new features of the latest version can be divided into four key themes \autocite{MWST2018}:

\begin{itemize}
	\item Hybrid cloud
	\item Security
	\item Application platform
	\item Hyper-converged infrastructure
\end{itemize}

Those four themes will be examined thoroughly in Chapter \ref{ch:stand-van-zaken}.

\section{\IfLanguageName{dutch}{Onderzoeksvraag}{Research question}}
\label{sec:onderzoeksvraag}
What are the advantages and disadvantages of a migration from Windows Server 2016 to Windows Server 2019 in a business environment?

\subsection{Sub-research question}

\begin{itemize}
	\item What are the differences between the Windows, Server Core and Nano Server base container images of Windows Server 2019?
	\item Can SAP be migrated from an existing Windows Server 2016 to Windows Server 2019 in a business environment?
	\item How can the new features of Windows Server 2019 be leveraged in the migrated infrastructure? 
\end{itemize}

\section{\IfLanguageName{dutch}{Onderzoeksdoelstelling}{Research objective}}
\label{sec:onderzoeksdoelstelling}

The expected result of this bachelor's thesis is to demonstrate the advantages of utilizing Windows Server 2019 over Windows Server 2016, proving that migration is a long-term investment towards the future for any organization. 
It will also compare the base container images of the \acrshort{os} and show that a migration of a SAP environment to Windows Server 2019 can be done. 
This because the  requirements between the latest version of both the \acrshort{os} and the SAP Kernel have been met. 

\section{\IfLanguageName{dutch}{Opzet van deze bachelorproef}{Structure of this bachelor's thesis}}
\label{sec:opzet-bachelorproef}

The remainder of this bachelor's thesis is structured as follows:

In Chapter ~\ref{ch:stand-van-zaken}, an overview is given of the state of the art within the research domain, based on a literature study.
\\

In Chapter ~\ref{ch:methodologie}, a methodology is explained, and the research techniques used to formulate an answer to the research questions are discussed.
\\

In Chapter ~\ref{ch:toekomstvisie}, the further development of Windows Server over the years to come will be analysed. 
\\

In Chapter~\ref{ch:conclusie}, finally, the conclusion is given, and an answer is formulated to the research questions. This will also provide an impetus for future research within this domain.
\\
