%%=============================================================================
%% Samenvatting
%%=============================================================================
%%---------- Nederlandse samenvatting -----------------------------------------
\IfLanguageName{english}{%
\selectlanguage{dutch}
\chapter*{Samenvatting}
Uit een studie is gebleken dat in 2016, 17.9\% en 45.4\% van de servers, in het Spiceworks-netwerk, respectievelijk Windows Server 2003 en Windows Server 2008 gebruikten. \autocite{Tsai2016} 
Deze worden niet langer ondersteund door Microsoft en de uitgebreide ondersteuning voor de laatste loopt ten einde januari 2020. 
Dit betekent dat een grootschalige update naar een nieuwere versie een vereiste wordt voor een aanzienlijk aantal organisaties. 
In deze bachelorproef zijn de voor- en nadelen van een migratie naar Windows Server 2019 onderzocht. 
Er is rekening gehouden met de vier hoofdthema's waarin de nieuwe functies zijn onderverdeeld. 
De migratie tussen de twee versies wordt vervolgens uitgevoerd aan de hand van een in-place upgrade en een side-by-side migratie, met Windows Server 2016 als startpunt. 
Hieruit bleek dat, hoewel een in-place upgrade in een eerste opzicht de eenvoudigste methode is, dit niet altijd de beste is. 
De voordelen van een side-by-side migratie, op vlak van veiligheid en performantie, verantwoorden op lange termijn de additionele inspanning die hiermee gemoeid is.
Aangezien dit \acrshort{os} de basis vormt voor vele applicaties, is ook een migratie van een SAP-omgeving uitgevoerd. 
De SAP Kernel heeft reeds ondersteuning voor Windows Server 2019. 
Een migratie van de gangbare softwareproducten die SAP aanbiedt, is dus mogelijk.
Dit vereist echter bijkomend onderzoek, zoals beschreven in deze bachelorproef. 
Opvallend is dat voor de migratie van andere softwarepakketten eveneens aanvullend onderzoek moet worden verricht. 
Dit om, net als bij Windows Server 2019 en SAP, ervoor te zorgen dat alles voldoet aan de kwaliteitseisen die een organisatie stelt.
Hierna werden de verschillende beschikbare base container images geanalyseerd. 
Door een verschil in grote en performantie ging de keuze naar de meest recente versie. 
Dit door de additionele functies, die doorheen deze bachelorproef besproken worden, en de toevoeging van de nieuwe Windows base container image.
Tot slot wordt er gekeken naar de toekomst van Windows Server. 
Voornamelijk naar het gebruik van de cloud en de nieuwe functionaliteiten die hiermee gepaard gaan in \acrlong{wac}. 

\selectlanguage{english}
}{}
%%---------- Engelse samenvatting -----------------------------------------------------
\chapter*{\IfLanguageName{dutch}{Samenvatting}{Abstract}}
A study showed that in 2016, 17.9\% and 45.4\% of servers, in the Spiceworks network, respectively used Windows Server 2003 and Windows Server 2008. \autocite{Tsai2016} 
These are no longer supported by Microsoft and the extended support for the latter runs until the end of January 2020. 
This means that a large-scale update to a newer version is becoming a requirement for many organizations. 
In this bachelor's thesis, the advantages and disadvantages of migrating to Windows Server 2019, are investigated. 
The four main themes, in which the new features of Windows Server 2019 are divided, were taken into consideration. 
The migration between both versions is then carried out according to the in-place upgrade and side-by-side migration method, starting from Windows Server 2016. 
This showed that although an in-place upgrade is in first respect the simplest method, it is not always the best. 
The advantages of a side-by-side migration in terms of safety and long-term performance justify the additional effort involved.
Since for many organizations, this \acrshort{os} is the basis for their applications, a migration of a SAP environment is also performed. 
The SAP Kernel supports Windows Server 2019. 
A migration of all the trivial software solutions offered by SAP is possible.
However, this requires additional research, as described in this bachelor's thesis.
It is striking that for the migration of other software solutions supplementary research needs to be carried out, as well as additional and rigorous testing. 
This, like done in this bachelor's thesis for Windows Server 2019 and SAP, to ensure that everything meets the quality demands of an organization.
After this, the different available base container images were analysed. 
This showed that because of a difference in size and performance, the choice went to the most recent version. 
This due to the additional functions, which are discussed during this bachelor's thesis and the addition of the new Windows base container image.
Finally, a look is taken at the future of Windows Server. 
Especially at the usage of the cloud and the new functionalities that are paired with this in \acrlong{wac} is taken. 
