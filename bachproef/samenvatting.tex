
%%=============================================================================
%% Samenvatting
%%=============================================================================
% - Context: waarom is dit werk belangrijk?
% - Nood: waarom moest dit onderzocht worden?
% - Taak: wat heb je precies gedaan?
% - Object: wat staat in dit document geschreven?
% - Resultaat: wat was het resultaat?
% - Conclusie: wat is/zijn de belangrijkste conclusie(s)?
% - Perspectief: blijven er nog vragen open die in de toekomst nog kunnen
%    onderzocht worden? Wat is een mogelijk vervolg voor jouw onderzoek?


%%---------- Nederlandse samenvatting -----------------------------------------
\IfLanguageName{english}{%
\selectlanguage{dutch}
\chapter*{Samenvatting}
Uit een onderzoek \autocite{Tsai2016} bleek dat, in 2016, 17.9\% en 45.4\% van de servers, in het Spiceworks netwerk, gebruik maakte van respectievelijk Windows Server 2003 en Windows Server 2008. Windows Server 2003 is officieel niet meer ondersteund door Microsoft en de uitgebreide ondersteuning voor Windows Server 2008 loopt ten einde januari 2020. Dit betekend dat een grootschaalse update naar een nieuwere versie niet enkel interessant, maar ook een vereiste aan het worden is. In deze bachelorproef werden de voor- en nadelen van een migratie naar de laatste versie van het besturingssysteem, Windows Server 2019, onderzocht. Hierbij werd rekening gehouden met de vier hoofdthema's die de additionele functies van Windows Server 2019 omvatten. Vervolgens wordt de migratie tussen de twee versies uitgevoerd op twee manieren uitgevoerd: 'in-place upgrade' en 'full system migration'. 
Hieruit bleek dat %TODO Conclusie
Verder werd ook onderzocht in welke mate een SAP omgeving gemigreerd kon worden, dit op vraag van de opdrachtgever, Delaware. %TODO SAP
%TODO Perspectief

\selectlanguage{english}
}{}
%%---------- Engelse samenvatting -----------------------------------------------------
\chapter*{\IfLanguageName{dutch}{Samenvatting}{Abstract}}

A study \autocite{Tsai2016} showed that, in 2016, 17.9\% and 45.4\% of the servers, in the Spiceworks network, used Windows Server 2003 and Windows Server 2008 respectively. Windows Server 2003 is officially no longer supported by Microsoft and the extensive support for Windows Server 2008 runs until the end of January 2020. This means that a large scale update to a newer version is not only interesting, but also becoming a requirement. In this bachelor's thesis, the pros and cons of migrating to the latest version of the operating system, Windows Server 2019, were investigated. The four main themes that include the additional functions of Windows Server 2019 were taken into account. The migration between  two versions is then carried out in two ways: In-place upgrade' and 'full system migration'. 
This showed that %TODO Conclusion
The extent to which an SAP environment could be migrated was also investigated, at the request of the client, Delaware. %TODO SAP
%TODO Perspective