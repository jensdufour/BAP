% !TeX spellcheck = en_GB
%===============================================================================
% LaTeX sjabloon voor de bachelorproef toegepaste informatica aan HOGENT
% Meer info op https://github.com/HoGentTIN/bachproef-latex-sjabloon
%===============================================================================
\documentclass{bachproef-tin}
\usepackage{hogent-thesis-titlepage} % Titelpagina conform aan HOGENT huisstijl
\usepackage{adjustbox}
\usepackage{graphicx}
\usepackage{caption}
\usepackage{subcaption}
\usepackage{listings}
\usepackage{hyperref}
\usepackage[section]{placeins}

% !TeX spellcheck = en_GB
\newacronym{eol}{EOL}{End Of Life}
\newacronym{it}{IT}{Information Technology}
\newacronym{os}{OS}{Operating System}
\newacronym{hr}{HR}{Human Resources}
\newacronym{vpn}{VPN}{Virtual Private Network}
\newacronym{atp}{ATP}{Advanced Threat Protection}
\newacronym{sdn}{SDN}{Software Defined Networking}
\newacronym{vm}{VM}{Virtual Machine}
\newacronym{wdatp}{WDATP}{Windows Server Advanced Threat Protection}
\newacronym{roi}{ROI}{Return on Investment}
%%---------- Documenteigenschappen ---------------------------------------------
% De titel van het rapport/bachelorproef
\title{The migration process and advantages of Windows Server 2019}
% Je eigen naam
\author{Jens Du Four}
% De naam van je promotor
\promotor{Ludwig Stroobant}

% De naam van je co-promotor.
\copromotor{Glenn Verschuere}
% Indien je bachelorproef in opdracht van een bedrijf
\instelling{Delaware}
% Academiejaar
\academiejaar{2018-2019}
% Examenperiode
\examenperiode{2}
%===============================================================================
% Inhoud document
%===============================================================================
\begin{document}	
%---------- Taalselectie -------------------------------------------------------
\selectlanguage{english}
%---------- Titelblad ----------------------------------------------------------
\inserttitlepage
%---------- Samenvatting, voorwoord --------------------------------------------
\usechapterimagefalse
% !TeX spellcheck = en_GB
%%=============================================================================
%% Voorwoord
%%=============================================================================

\chapter*{\IfLanguageName{dutch}{Woord vooraf}{Preface}}
\label{ch:voorwoord}

Throughout the following pages, I will research the question: 'What are the advantages and disadvantages of a migration from Windows Server 2016 to Windows Server 2019 in a business environment?'. 
This bachelor's thesis was written in the context of my graduation at the University College Ghent where I studied Applied Computer Sciences and by order of delaware, where I researched and wrote my bachelor's thesis from February 2019 till June 2019.

In consultation with my promoter, Ludwig Stroobant, and my co-promoter, Glenn Verschuere, the research question was formulated as the basis. 
For the duration of this assignment, they were always there to answer the questions I had regarding the research. 
In addition, I would like to thank Kasper Heyndrickx. 
His technical knowledge of SAP was of immense importance for this bachelor's thesis.

I would like to thank them. 
Without them, this would not have been possible.
Furthermore, I would like to give a special thanks to delaware and especially to Tomas Castro and Jonas Decoster. 
Without their effort, I would never have had the opportunity to do this research from their branch in Harbin, China.
I would also like to thank Henry Hao and Cici Chen for their hospitality and guidance during my stay.
It was an experience I will never forget and always be grateful for. 

Finally, I would like to thank my parents. Their wisdom and motivating words have helped me to bring this bachelor's thesis to a successful conclusion.\\
I wish you a lot of reading pleasure.

Jens Du Four

Ghent, 31/05/2019
%%=============================================================================
%% Samenvatting
%%=============================================================================
%%---------- Nederlandse samenvatting -----------------------------------------
\IfLanguageName{english}{%
\selectlanguage{dutch}
\chapter*{Samenvatting}
Uit een studie is gebleken dat in 2016 17.9\% en 45.4\% van de servers, in het Spiceworks-netwerk, respectievelijk Windows Server 2003 en Windows Server 2008 gebruikten. \autocite{Tsai2016} 
Deze worden officieel niet langer ondersteund door Microsoft en de uitgebreide ondersteuning voor de laatste loopt ten einde januari 2020. 
Dit betekent dat een grootschalige update naar een nieuwere versie een vereiste wordt voor een groot aantal organisaties. 
In deze bachelorproef zijn de voor- en nadelen van een migratie naar Windows Server 2019 onderzocht. 
Er is rekening gehouden met de vier hoofdthema's waarin de niewe functies van zijn onderverdeeld. 
De migratie tussen de twee versies wordt vervolgens uitgevoerd aan de hand van de in-place en de volledige migratiemethode, met Windows Server 2016 als startpunt. 
Hieruit bleek dat hoewel een in-place migratie in een eerste opzicht de eenvoudigste methode is, dit niet altijd de beste is. 
De voordelen van een volledige migratie op vlak van veiligheid en performantie op lange termijn verantwoorden de additionele inspanning die hiermee gemoeid is.
Aangezien dit \acrshort{os} de basis vormt voor vele applicaties, is ook een migratie van een SAP-omgeving uitgevoerd. 
%TODO SAP Conclusion
Hierna, werden de verschillende beschikbare Windows Base Container Images geanalyseerd. 
Door een verschil in grote en performantie ging de keuze naar de meest recente versie. 
Dit door de additionele functies, die doorheen deze bachelorproef besproken worden, en de toevoeging van de nieuwe Windows Base Container Image.
Tot slot wordt er gekeken de toekomst van Windows Server. 
Hier wordt gekeken naar het gebruik van de cloud, en de nieuwe functionaliteiten die hiermee gepaard gaan in \acrlong{wac}. 
Opvallend is dat voor de migratie van andere softwarepakketten aanvullend onderzoek moet worden verricht, evenals een testomgeving. 
Dit om, net als bij Windows Server 2019 en SAP, ervoor te zorgen dat alles voldoet aan de eisen die een organisatie stelt.

\selectlanguage{english}
}{}
%%---------- Engelse samenvatting -----------------------------------------------------
\chapter*{\IfLanguageName{dutch}{Samenvatting}{Abstract}}
A study showed that in 2016, 17.9\% and 45.4\% of servers, in the Spiceworks network, respectively used Windows Server 2003 and Windows Server 2008. \autocite{Tsai2016} 
These are officially no longer supported by Microsoft and the extended support for the latter runs until the end of January 2020. 
This means that a large-scale update to a newer version is becoming a requirement for many organizations. 
In this bachelor's thesis the advantages and disadvantages of migrating to Windows Server 2019 are investigated. 
The four main themes, in which the new features of Windows Server 2019 are divided, were taken into account. 
The migration between both versions is then carried out according to the in-place and full migration method, starting from Windows Server 2016. 
This showed that although an in-place migration is in a first respect the simplest method, it is not always the best. 
The advantages of a complete migration in terms of safety and long-term performance justify the additional effort involved.
Since for many organizations this \acrshort{os} is the basis for their applications, a migration of an SAP environment is also performed. 
%TODO SAP Conclusion
After this, the different available Windows Base Container Images were analysed. 
This showed that because of a difference in size and performance, the choice went to the most recent version. 
This due to the additional functions, which are discussed during this bachelor's thesis, and the addition of the new Windows Base Container Image.
Finally, a look is taken at the future of Windows Server. 
Here a look at the usage of the cloud, and the new functionalities that are paired with this in \acrlong{wac}, is taken. 
It is striking that for the migration of other software solutions supplementary research needs to be carried out, as well as additional and rigorous testing. 
This, like done in this bachelor's thesis for Windows Server 2019 and SAP, to ensure that everything meets the requirements of an organization.
%---------- Inhoudstafel -------------------------------------------------------
\pagestyle{empty} % Geen hoofding
\tableofcontents  % Voeg de inhoudstafel toe
\cleardoublepage  % Zorg dat volgende hoofstuk op een oneven pagina begint
\pagestyle{fancy} % Zet hoofding opnieuw aan
%---------- Lijst figuren, afkortingen, ... ------------------------------------
\listoffigures
\listoftables
% Als je een lijst van afkortingen of termen wil toevoegen, dan hoort die hier thuis.
\printnoidxglossaries
%---------- Kern ---------------------------------------------------------------
% !TeX spellcheck = en_GB
%%=============================================================================
%% Inleiding
%%=============================================================================
\chapter{\IfLanguageName{dutch}{Inleiding}{Introduction}}
\label{ch:inleiding}
Windows Server is an \acrfull{os} that is widely used by organizations all over the world. 
Some of them with only a handful of employees to corporations that have a couple of thousand at their disposal, in multiple geographic locations. 
To keep up in the fast-paced world that is \acrfull{it}, updates are a necessary part of the daily operation, though they do not always present themselves at the most convenient time. 
\acrfull{sac} updates which can be scheduled to \acrfull{ltsc} updates, that could require entire systems to be taken offline for an extended duration. 
This can bring tremendous damage to the business value of the organization since \acrshort{it} has become a core business for many of them. 
Therefore, asking how this can be done as efficient and cost-effective as possible is important before considering a migration to the latest version. 

\section{\IfLanguageName{dutch}{Probleemstelling}{Problem Statement}}
\label{sec:probleemstelling}
Since the release of Windows Server 2019, the latest version, many organizations are willing to look into a migration from previous versions of the \acrshort{os} in the nearby future. 
One of these organizations, Delaware, wants to research how this migration would take place, starting from Windows Server 2016, and how the migration can be achieved in an efficient and cost-effective way. 

\begin{table}[htb!]
	\centering
	\begin{adjustbox}{width=1\textwidth}
		\begin{tabular}{l|l|l|ll}
			Products Released & Life Cycle Start Date & Mainstream Support End Date & Extended Support End Date &\\
			\hline
			Windows Server 2016 Standard & 15/10/2016 & 11/01/2022 & 12/01/2027 &\\
			Windows Server 2016 Datacenter & 15/10/2016 & 11/01/2022 & 12/01/2027 &\\
		\end{tabular}
	\end{adjustbox}
	\caption[\acrshort{eol} \acrshort{ws}2016]{\acrshort{eol} of Windows Server 2016}
	\scriptsize	
	Adapted from \cite{MicrosoftEOL2019}
	\label{tab:EOL2016}
\end{table}

With the ending of mainstream support for Windows Server 2016 scheduled at the start of 2022, as seen in Table ~\ref{tab:EOL2016}, it is best to research the migration to the latest version well in advance. 
Especially when considering that organizations running \acrfull{eol} \acrshort{os}s are very common. \\
In an article written by \textcite{Tsai2016}, it is shown that the market share, in the Spiceworks network, of Windows Server 2003 was at 17.9\% and that of Windows Server 2008 was at 45.4\%, in 2016. 
The official support for both of these versions has already ended. The extended support for the latter will be ending at the start of 2020. 

\begin{table}[ht]
	\centering
	\begin{adjustbox}{width=1\textwidth}
		\begin{tabular}{l|l|l|ll}
			Products Released & Life Cycle Start Date & Mainstream Support End Date & Extended Support End Date &\\
			\hline
			Windows Server 2019 Essentials & 13/11/2018 & 09/01/2024 & 09/01/2029 &\\
			Windows Server 2019 Standard & 13/11/2018 & 09/01/2024 & 09/01/2029 &\\
			Windows Server 2019 Datacenter & 13/11/2018 & 09/01/2024 & 09/01/2029 &\\
		\end{tabular}
	\end{adjustbox}
	\caption[\acrshort{eol} \acrshort{ws}2019]{\acrshort{eol} of Windows Server 2019}
	\scriptsize	
	Adapted from \cite{MicrosoftEOL2019}
	\label{tab:EOL2019}
\end{table}

Those deadlines combined with the \acrshort{eol} of Windows Server 2016, make a strong case to look into the migration to the latest version, which can offer enhanced security and additional features. 
The life cycle of the latest version, Windows Server 2019, started in November 2018 and it will continue to receive mainstream support until January 2024. 
The extended support is guaranteed until January 2029, as seen in Table ~\ref{tab:EOL2019}.\\
This makes it a viable successor for the \acrshort{os} that makes up a large percent of the server infrastructure used in the world. 
This bachelor's thesis attempts to show that the benefits will outweigh the disadvantages that are paired with a migration. 
The new features of the latest version can be divided into four key themes:

\begin{itemize}
	\item Hybrid cloud
	\item Security
	\item Application platform
	\item Hyper-converged infrastructure
\end{itemize}

Those four themes will be examined very thoroughly in Chapter \ref{ch:stand-van-zaken}.

\section{\IfLanguageName{dutch}{Onderzoeksvraag}{Research question}}
\label{sec:onderzoeksvraag}
What are the advantages and disadvantages of a migration from Windows Server 2016 to Windows Server 2019 in a business environment?

\subsection{Sub-research question}

\begin{itemize}
	\item What are the differences between the Windows, ServerCore and Nano Server Base Container Images of Windows Server 2019?
	\item Can SAP be migrated from an existing Windows Server 2016 to Windows Server 2019 in a business environment?
	\item How can the new features of Windows Server 2019 be leveraged in the migrated infrastructure? 
\end{itemize}

\section{\IfLanguageName{dutch}{Onderzoeksdoelstelling}{Research objective}}
\label{sec:onderzoeksdoelstelling}

The expected result of this bachelor's thesis is to demonstrate the advantages of utilizing Windows Server 2019 over Windows Server 2016, proving that migration is a long-term investment towards the future. 
It will also compare the Base Container Images of the \acrshort{os}: Windows, Windows Server Core and Windows Nano Server. \\
However, migrating an SAP environment will not yet be recommended due to the new nature of Windows Server 2019. 
This due to requirements that have not yet been met between the \acrshort{os} and the SAP Kernel. 

\section{\IfLanguageName{dutch}{Opzet van deze bachelorproef}{Structure of this bachelor thesis}}
\label{sec:opzet-bachelorproef}

The remainder of this bachelor's thesis is structured as follows:
In Chapter ~\ref{ch:stand-van-zaken}, an overview is given of the state of affairs within the research domain, based on a literature study.\\
In Chapter ~\ref{ch:methodologie}, a methodology is explained and the research techniques used to formulate an answer to the research questions are discussed.\\
In Chapter ~\ref{ch:toekomstvisie}, the further development of Windows Server over the years to come will be analysed. \\
In Chapter~\ref{ch:conclusie}, finally, the conclusion is given and an answer is formulated to the research questions. This will also provide an impetus for future research within this domain.\\

% !TeX spellcheck = en_GB
\chapter{\IfLanguageName{dutch}{Stand van zaken}{State of the art}}
\label{ch:stand-van-zaken}

% Tip: Begin elk hoofdstuk met een paragraaf inleiding die beschrijft hoe
% dit hoofdstuk past binnen het geheel van de bachelorproef. Geef in het
% bijzonder aan wat de link is met het vorige en volgende hoofdstuk.

This chapter contains the state of affairs. As mentioned in the introduction this part of the bachelor's thesis will be used to get an in-depth understanding of the four key themes that incorporate the changes in Windows Server 2019. First, for the key themes mentioned before, a basic understanding of what they are will be established together with how these have been implemented in the latest version of the \acrshort{os}.

%Dit hoofdstuk bevat je literatuurstudie. De inhoud gaat verder op de inleiding, maar zal het onderwerp van de bachelorproef *diepgaand* uitspitten. De bedoeling is dat de lezer na lezing van dit hoofdstuk helemaal op de hoogte is van de huidige stand van zaken (state-of-the-art) in het onderzoeksdomein. Iemand die niet vertrouwd is met het onderwerp, weet er nu voldoende om de rest van het verhaal te kunnen volgen, zonder dat die er nog andere informatie moet over opzoeken \autocite{Pollefliet2011}.
%\textcite{Knuth1998} schreef een van de standaardwerken over sorteer- en zoekalgoritmen. Experten zijn het erover eens dat cloud computing een interessante opportuniteit vormen, zowel voor gebruikers als voor dienstverleners op vlak van informatietechnologie~\autocite{Creeger2009}.

% Pas na deze inleidende paragraaf komt de eerste sectiehoofding.
\section{Hybrid cloud}

Hybrid Cloud is a topic that has gained more momentum over the past years. This makes it a consistent topic of interest for organizations and makes it one of the keystone themes in Windows Server 2019. \autocite{MWST2018} It is with this in mind that it is an essential part of the bachelor's thesis. It is especially beneficial to know the advantages on this aspect of Windows Server since the latest version and how these improve workflow and how these can be leveraged by organizations, in particular, Delaware.

\subsection{Types of cloud solutions}

The National Institute of Standards and Technology differentiates four types of clouds \autocite{Mell2011}:
\begin{itemize}
	\item Private Cloud
	\item Community Cloud
	\item Public Cloud
	\item Hybrid Cloud
\end{itemize}	

%TODO Usage of types of clouds in a business environment
%Poll op LinkedIn

\subsubsection{Private cloud}
A private cloud is an internal infrastructure in the cloud that is designated for usage by a single organization, that can consist of multiple clients, albeit in the same organization. It is only accessible inside a private internal network or over the Internet for a selected amount of users than for the public. It can also be known by other names such as internal or corporate cloud. Private clouds main advantage is the higher level of security and privacy they offer for organizations through the usage of in-house hosted infrastructure and additional company firewalls. The biggest disadvantage that comes with this added layer of security, is the responsibility that is given to the information technology team that manages the private cloud. This means, on top of the additional in-house hosting cost, that they require the same amount of man-hours that come with the management of a traditional datacenter. 
Still, the private cloud holds a great benefit compared to long-standing methods. As reported by IBM, which saved more than \$1.5 billion by reducing their number of datacenters from 115 to 5. This is due to the implementation of a private cloud. \autocite{Hofmann2010} 


\subsubsection{Community cloud}
When several organizations collaborate to meet the requirements that are demanded of the IT infrastructure, they are using a community cloud. This means that it can be managed in-house or by a third-party organization that operates inside the same community. This form of operation tackles one of the main problems of a private cloud, it shares the costs over multiple organizations. Since they are operating inside the same community, they share the same concerns and will be subjected to the same requirements that can be imposed by a governing instance. 
This advantage over private clouds is only a partial improvement. This reduction in cost thanks to the sharing of infrastructure means a reduction in security. Meaning that it is a viable alternative to  organizations that have some security concerns with the usage of a public cloud. 
\newline
An example of this is described in an article \autocite{Yao2014} about how small hospitals in China, not all of which can provide their own infrastructure, could utilize a community cloud. These grass-roots healthcare institutions, that are all within the same community, can share the cost and management of this cloud to provide an attractive hospital information solution to improve their service without the extensive cost nor the need for additional security concerns regarding confidential information in patient files.

\subsubsection{Public cloud}
Amazon Web Services, Oracle Cloud and Microsoft Azure are only some examples of public clouds. Most solutions are offered by organizations, like those mentioned above, who manage and operate their datacenter and provide access to their cloud via the Internet. This eliminates the cost that is associated with the management and responsibility of a private cloud, where the IT department is responsible, and thus significantly reduces the cost in same cases. Public cloud also provides the possibility for easy scalability and flexibility in comparison with private clouds, where the hardware needs to be available in-house. This makes it ideal for temporary solutions.
\newline
\textcite{Singh2012} concluded, in a comparison between the cost and security of private and public clouds over three years, that although security can be a real concern in the usage of public cloud it should not be ruled out immediately without fully analysing the requirements of an organization, this with keeping in mind the major investment that comes paired with the usage and implementation of a private cloud.  
The different obstacles that securing a public cloud has are also addressed by \textcite{Ren2012}, in which there is a call for additional research about the subject to fully take advantage of the revelation that cloud computing is. 

\subsubsection{Hybrid cloud}
A hybrid cloud aims to be the solution for every business. It combines the higher level of security and privacy that is offered by the private cloud with easy scalability and flexibility that comes paired with the public cloud. In a hybrid cloud the organization manages a part of its cloud infrastructure in-house and a part out-house. As described in the book by \textcite{Sarna2010}, hybrid clouds enables large organizations to move their less sensitive information, like \acrfull{hr}, to the cloud. Thanks to the advantages of hybrid cloud their sensitive data, such as classified information about costumers or the organization, can remain in-house on private clouds or even on-premise for an additional layer of security. The connection between both the public and private cloud part of the hybrid cloud is generally accomplished through a \acrfull{vpn}.

%\subsubsection{Conclusion}
%TODO Is een conclusie hier vereist?

\subsection{Hybrid cloud in Windows Server 2019}
%TODO Windows Admin Center
\clearpage

\section{Security}
With more than 53.000 reported incidents and 2.216 confirmed data breaches \autocite{Verizon2018}, security has become an essential part of \acrshort{it}. The importance of security directly translates into Windows Server 2019 through various parts of the \acrshort{os} that have either been reviewed to make them more resilient and accessible or new features which have been added to further improve security. In the following section all the major elements will be discussed, distributed among three subsections:
\begin{itemize}
	\item Windows Defender \acrfull{atp}
	\item Security with \acrfull{sdn}
	\item Shielded Virtual Machines
\end{itemize}

\subsection{Windows Defender \acrfull{atp}}
In a research study done by \textcite{Musto2017} \acrfull{wdatp} was scrutinized. When the endpoint security solution is implemented in Windows Server 2019, it requires additional licensing. Still, the efficiency with which it tackles security problems resulted in a 53\% \acrfull{roi}. The research reported that \acrshort{wdatp} reduced the risk of a breach by 40\%. It even enabled them to identify threats faster and resolve them in a more efficient fashion. In conclusion, the replacement of previous solutions with \acrlong{wdatp} reduced costs and made security teams more efficient. With the advent of Windows Server 2019, additional features have been added to \acrlong{wdatp} to ensure the safety of organizations in the years to come. The different components of \acrshort{wdatp} are described in the picture below. Those will not be individually reviewed here as this subject alone offers enough content for another bachelor's thesis.

\begin{figure}[hbt!]
	\centering
	\includegraphics[width=\textwidth,height=6cm,keepaspectratio=true]{img/Windows-Defender-ATP.png}
	\caption[Components of \acrshort{wdatp}]{The different components of \acrlong{wdatp}. \autocite{Aslaner2018}}
	\label{fig:WDATPT2018}
\end{figure}
%\subsubsection{Attack surface reduction}
%\subsubsection{Next generation protection}
%\subsubsection{Endpoint detection and response}
%\subsubsection{Auto investigation and remediation}
%\subsubsection{Security posture}
%\subsubsection{Advanced hunting}

\subsection{Security with \acrfull{sdn}}
As described by \textcite{Shin2016}, \acrfull{sdn} is a state-of-the-art technology that enables developers to design advanced networks effortless. In Windows Server 2019 their have also been new development in this field, these can be boiled down to four features. The discussion of these will be constrained.
\begin{itemize}
	\item Encrypted networks
	\item Firewall auditing
	\item Virtual network peering
	\item Egress metering
\end{itemize} 

\subsubsection{Encrypted networks}
Encrypted networks, or more specifically virtual encrypted networks, enable the encryption of network traffic between different \acrlong{vm}s. It does this through the usage of \acrfull{dtls}. \acrshort{dtls} was built as close to \acrshort{tls} as possible \autocite{Modadugu2003}, which makes it ideal at securing connections. However, for the connection between \acrlong{vm}s \acrshort{dtls} is preferred, since these connections are delay sensitive. 
\subsubsection{Firewall auditing}
One of the new features in Windows Server 2019 is \acrshort{sdn} Firewall auditing. This means that the administrator is enabled to see if every part of his Firewall is as secure as is initially thought. When this feature is enabled every data stream that gets processed by the \acrshort{sdn} firewall, gets recorded. The logs can than be used in troubleshooting or archived for analysis. These can also be processed by using tools such as Power BI.
\subsubsection{Virtual network peering}
Virtual network peering allows you to combine two individual virtual networks. A coherent connection is made that represents itself as an individual one. This means the connection between both networks can be routed through the infrastructure backbone, this in term means that there is no need for a public gateway. Providing a more secure connection, while no additional downtime is accumulated when peering the networks.

\subsubsection{Egress metering}


\subsection{Shielded Virtual Machines}

\section{Application platform}
%\subsection{Linux containers on Windows}
%\subsection{Building Support for Kubernetes}
%\subsection{Container improvements}
%\subsection{Encrypted Networks}
%\subsection{Network performance improvements for virtual workloads}
%\subsection{Low Extra Delay Background Transport}
%\subsection{Windows Time Service}
%\subsection{High performance SDN gateways}
%\subsection{New Deployment UI and Windows Admin Center extension for SDN}
%\subsection{Persistent Memory support for Hyper-V VMs}

\section{Hyper-converged infrastructure}
% !TeX spellcheck = en_GB
%%=============================================================================
%% Methodologie
%%=============================================================================

\chapter{\IfLanguageName{dutch}{Methodologie}{Methodology}}
\label{ch:methodologie}
% Tip: Begin elk hoofdstuk met een paragraaf inleiding die beschrijft hoe
% dit hoofdstuk past binnen het geheel van de bachelorproef. Geef in het
% bijzonder aan wat de link is met het vorige en volgende hoofdstuk.

%% TODO: Hoe ben je te werk gegaan? Verdeel je onderzoek in grote fasen, en
%% licht in elke fase toe welke stappen je gevolgd hebt. Verantwoord waarom je
%% op deze manier te werk gegaan bent. Je moet kunnen aantonen dat je de best
%% mogelijke manier toegepast hebt om een antwoord te vinden op de
%% onderzoeksvraag.

\lipsum[21-25]


% !TeX spellcheck = en_GB
%%=============================================================================
%% Toekomstvisie
%%=============================================================================
\chapter{\IfLanguageName{dutch}{Toekomstvisie}{Future vision}}
\label{ch:toekomstvisie}
This final chapter will focus on the public, private and on-premise solutions that Microsoft offers. 
Especially, how these are used as an extension of each other to build a modern infrastructure. 
It is not the first time that Microsoft Azure has been mentioned in this bachelor's thesis. 
It has been extensively used for the deployment of different proof of concept environments, as well as being mentioned in Chapter \ref{ch:stand-van-zaken}. 
The future of this environment will be discussed first. 
Afterwards, the future of their private cloud solution, Microsoft Azure Stack, will be examined. 
Furthermore, what can be expected from their on-premise solution, Windows Server 2019, in the future will be discussed. 
In particular the \acrlong{wac}, will be reviewed. 
The discussion of the different aspects exclusively has the intention to give the reader an idea of what the future can hold. 
It will be done cursory because all the different sections in this chapter contain sufficient content for their own bachelor's thesis. 
Finally, how these can be used in unison to form the infrastructure of tomorrow, as described in Figure \ref{fig:Azure_FullCircle}, will be concluded.

\begin{figure}[h]
	\captionsetup{width=0.6\linewidth}
	\includegraphics[width=0.7\linewidth]{img/Toekomstvisie/Azure1.png}
	\centering
	\caption[Modern infrastructure]{Bridging on-premise and cloud solutions for a modern infrastructure}
	\scriptsize	
	Adapted from \cite{Singh2019}
	\label{fig:Azure_FullCircle}
\end{figure}

\clearpage

\section{Microsoft Azure}
As previously mentioned, Microsoft Azure, is the public cloud solution Microsoft offers. 
There are currently more than fifty different global Azure regions and their network keeps expanding. 
It offers a wide array of solutions, some of which, like SAP on Azure, have been used in this bachelor's thesis. 
These are continuously updated to be on-par with the latest developments in \acrshort{it} such as \acrfull{ai} and blockchain. 
Microsoft also offers a selection of these services in China. 
These services can be managed from the Windows, macOS  and Linux \acrshort{os} through Azure Powershell. 
Alternatively they can be managed through the Azure Portal as can be seen in Figure \ref{fig:Azure_Portal}.

\begin{figure}[h]
	\captionsetup{width=0.8\linewidth}
	\includegraphics[width=0.9\linewidth]{img/Toekomstvisie/Azure0.png}
	\centering
	\caption[Azure Portal]{Microsoft Azure Portal landing page}
	\label{fig:Azure_Portal}
\end{figure}
Considering that the public cloud revenue is expected to grow 17.3\% in 2019, this is certainly one of the technologies that will play an important role in the future.\autocite{Ng2018}

\section{Microsoft Azure Stack}
As has been mentioned in Chapter \ref{ch:stand-van-zaken}, security concerns are one of the major bottlenecks when choosing for a public cloud. 
While most modern applications are being moved into the public cloud, some of these have additional requirements that can only be met by the usage of an on-premise solution. 
As an alternative, many organizations chose the additional security of a private cloud, a service that Microsoft offers through Microsoft Azure Stack. 
It allows organizations to run these applications in an on-premise environment, it brings the innovation of cloud computing to an on-premise environment.
It can be seamlessly integrated into Microsoft Azure to provide a hybrid cloud environment, which is an essential part of this bachelor's thesis, and of which the advantages have already been thoroughly discussed.

\section{\acrfull{wac}}
The final part of the puzzle that is going to be discussed is the \acrlong{wac}. 
It is an essential part of Windows Server 2019 and provides the resources to manage your on-premise infrastructure with the ease and efficiency that comes with a cloud solution. 
It also provides everything needed to interlink your public cloud, hosted on Microsoft Azure, with the on-premise infrastructure. 
Some features available, like Azure Site Recovery and Azure Backup, have already been discussed in Chapter \ref{ch:stand-van-zaken}, but it is important to note that a lot of new features are going to be added to the \acrlong{wac} over the following months.\autocite{Singh2019} 
\\
Features like Azure File Sync, file shares in the Azure public cloud, and Azure Migration, which will make the migration from Windows Server to Microsoft Azure even more fluent, have already been announced. 
\acrlong{wac} also, provides support for manufacturers extensions through its own \acrfull{sdk}. 	

\section{Conclusion}
As has been described in Figure \ref{fig:Azure_FullCircle}, Microsoft aims for a migration to the cloud. 
They want to provide an environment that distributes workloads dynamically, either between on-premise or cloud solutions. 
Windows Server 2019 was designed with this in mind, this is also explains why hybrid cloud and \acrshort{hci} are key themes in the \acrshort{os}. 
It is an \acrshort{os} that forms a bridge between the on-premise and cloud environments. 
It does this through the implementation of features, like \acrfull{s2d} and \acrfull{sdn}, which were previously only used in cloud environments. 
Microsoft aims to introduce new features several times a year that follow this ideology.
\\
They want to offer support to all of these components with a modern and future proof infrastructure in mind. 
Either by using a hybrid data centre, Windows Server 2019 combined with Microsoft Azure, or a hybrid cloud, Microsoft Azure combined with Microsoft Azure Stack.
% !TeX spellcheck = en_GB
%%=============================================================================
%% Conclusie
%%=============================================================================
\chapter{\IfLanguageName{dutch}{Conclusie}{Conclusion}}
\label{ch:conclusie}

% Trek een duidelijke conclusie, in de vorm van een antwoord op de onderzoeksvra(a)g(en).
% Wat was jouw bijdrage aan het onderzoeksdomein en
% hoe biedt dit meerwaarde aan het vakgebied/doelgroep? 
% Reflecteer kritisch over het resultaat. 
% Had je deze uitkomst verwacht? 
% Zijn er zaken die nog niet duidelijk zijn?
% Heeft het onderzoek geleid tot nieuwe vragen die uitnodigen tot verder onderzoek?

%%Research question
%What are the advantages and disadvantages of a migration from Windows Server 2016 to Windows Server 2019 in a business environment?
%%Sub-research question
%What are the differences between the Windows, ServerCore and Nano Server Base Container Images of Windows Server 2019?
%Can SAP be migrated from an existing Windows Server 2016 to Windows Server 2019 in a business environment?
%How can the new features of Windows Server 2019 be leveraged in the migrated infrastructure? 

The final conclusion of this bachelor's thesis is that a migration to Windows Server 2019 is achievable.
While a migration to Windows Server 2019 in an environment with a third-party application requires additional research, it was attainable for an SAP environment. 
The migration from Windows Server 2016 to Windows Server 2019 is also feasible for any organization running an \acrshort{eol} \acrshort{os}, as expected with the latest version of the \acrshort{os}.
This for either the general \acrshort{os} and the base container images.
Windows Server 2019 offers a wide array of improvements in terms of security, hybrid cloud, application platform and the \acrfull{hci}.
%It makes the organizations environment future-proof. 
The addition of the new Windows base container image provides the tools for automated \acrshort{ui} tests and its additional dependencies make it perfect for utilization with out-of-date packages. 
The Server Core base container image can be used to run typical Windows services. 
It offers application compatibility and has a wide array of built-in Windows roles and features. 
The Nano Server base container image was designed for 'born in the cloud' applications that provide an agile deployment and offer on-demand availability. 
The reduced footprint of the base container images without a drop of performance make the latest version the natural choice. 
The new features which are introduced can be leveraged through the usage of the \acrfull{wac}.
Organizations that are running \acrshort{eol} \acrlong{os}s should consider the migration to Windows Server 2019.
Organizations that are running \acrshort{eol} infrastructure could also consider a migration to the cloud, although this imposes additional research towards the advantages of a cloud solution in comparison to an in-house infrastructure.

%%=============================================================================
%% Bijlagen
%%=============================================================================
\appendix
\renewcommand{\chaptername}{Appendix} 
% TODO: Volgorde van Appendix controleren op het einde
%%---------- Onderzoeksvoorstel -----------------------------------------------
\chapter{Research proposal}
The subject of this bachelor's thesis is based on a research proposal that has been assessed in advance by the promoter. This proposal is included in this appendix.
% Verwijzing naar het bestand met de inhoud van het onderzoeksvoorstel
% !TeX spellcheck = en_GB
%---------- Introduction  ---------------------------------------------------------
\section{Introduction}\label{sec:introduction}
Windows Server is a well-known operating system among IT professional. With mayor organizations, like Infosys \autocite{S.Chauhan2015}, over the globe, have implemented some form of the operating system in their back-end. One of the tasks the OS often performs is keeping applications up-to-date in a Windows, however, it is not often described how to migrate the OS itself to a newer version. 
This does not mean that frequent updates are not important for both the addition of new features as well as the improvement of security, as this is becoming one of the big concerns in the 21st century. 
This research paper will go in depth about the advantages and disadvantages that a whole system migration to the newer version will bring with it. In specific, this paper will take a look at the migration from version 2016 to 2019, with as an optional requirement the migration of SAP, as this is applicable to the infrastructure of the client, Delaware.
The key purpose is to find a solution that is as efficient as possible to minimize possible downtime and to be applicable not only to the client, Delaware but also to other organizations that find them self in the same situation and want to migrate their back-end to the latest version.
\subsection{Research question}
What are the advantages and disadvantages of a migration from Windows Server 2016 to 2019 in a business environment?
\subsection{Sub-research question}
\begin{itemize}
	\item What are the differences between the standard, ServerCore and Nano version of Windows Server 2019?
	\item Can SAP be migrated from an existing Windows Server 2016 to 2019 in a business environment?
	\item How can the new features of Windows Server 2019 be leveraged in the migrated infrastructure? 
\end{itemize}
%---------- State of the art ---------------------------------------------------
\section{State-of-the-art}\label{sec:state-of-the-art}
Windows Server 2019 was built on the strong foundation of 2016 although it brings some new features \autocite{Gerend2018} in the mix. 
These can be boiled down to four key themes:
\begin{enumerate}
	\item Hybrid Cloud
	\item Security
	\item Application Platform
	\item Hyper-Converged Infrastructure (HCI)
\end{enumerate}
In the following section, the basic working of the above themes is discussed.
\subsection{Hybrid Cloud}
One of the new features in terms of Hybrid Cloud is the Server Core App Compatibility Feature on Demand \autocite{Pacquer2018}. This is an optional feature pack that was designed for Windows Server 2019 Server Core installations and version 1809 and can be applied to the system at all times. 
It significantly improves the application compatibility by adding binaries and packages and even with these added packages, the footprint of the machine will remain as small as possible. It achieves this by not implementing the Windows Desktop Experience GUI.
\subsection{Security}
Security has always been a fundamental part of an operating system. With over 53.000 reported incidents and 2.216 confirmed data breaches \autocite{Verizon2018}, security is without a doubt hot topic. This is heavily reflected in the newest edition of Windows Server with additions like Windows Defender Advanced Threat Protection, Security with Software Defined Networking and Shielded Virtual Machines. This research paper will also explore the additional safety mechanisms that were added.
\subsection{Application Platform}
At the heart of a server, applications can be found. Different applications allow one server to provide multiple services for end-users. There have been key improvements in this aspect as well. These improvements are mostly virtualization or security oriented. 
\subsection{Hyper-Converged Infrastructure}
This technology could easily come out of buzzword bingo however, it is perhaps the most exciting feature that has gotten improved support from Microsoft in the release of Windows Server 2019. HCI makes it effortless to scale up from two nodes to a hundred. This makes the technology intriguing for organizations with an in-house datacenter.
%---------- Methodology ------------------------------------------------------
\section{Methodology}\label{sec:methodology}
Extensive will research will be done before attempting the migration. The effective proof-of-concept will consist of infrastructure with Windows Server 2016 that will be migrated to Windows Server 2019. This will be done for the Windows Desktop Experience version as well as the ServerCore and Nano version. By carrying out this migration the advantages and disadvantages of the migration will become clear. New features from Windows Server 2019 will also be implemented in the migrated system. 
\\
Carrying out the migration on all three versions will make the limitations and features of the different versions very clear. 
To conclude an attempt will be made to migrate an existing SAP environment to our new and optimized Windows Server 2019 infrastructure.
\\
The goal at the end of the proof-of-concept is to make a reasoned decision between Windows Server 2016 or 2019 and the specific versions, Windows Desktop Experience, ServerCore or Nano.
%---------- Expected results ----------------------------------------------
\section{Expected results}\label{sec:anticipated_results}
It is expected that the proof-of-concept will demonstrate the advantages of the newer Windows Server and that the migration is an investment towards the future. With the new features and improvements, the Windows Desktop Experience will have the upper hand in comparison to the ServerCore and Nano version. However, migrating an SAP will not yet be recommended due to the new nature of Windows Server 2019. 
%---------- Expected conclusion ----------------------------------------------
\section{Expected conclusions}\label{sec:anticipated_conclusions}
It is expected to conclude that migration from Windows Server 2016 to 2019 is perfectly manageable however additional studies will be necessary to make sure that all software packages are already supported by the operating system.
%%---------- Windows Admin Center Azure integration --------------------------------------------------
\chapter{\acrfull{wac} Azure integration}
\label{WACAzure}
%TODO Change subcaptions
\begin{figure}[h]
	\begin{subfigure}{\textwidth}
		\includegraphics[width=0.9\linewidth]{img/WAC_Azure_1.png}
		\captionsetup{width=0.8\linewidth}
		\centering		
		\caption{Click next to upgrade}
		\label{fig:WACAzure1}
	\end{subfigure}
\end{figure}
\begin{figure}[h]\ContinuedFloat
	\begin{subfigure}{0.5\textwidth}
		\captionsetup{width=0.8\linewidth}
		\includegraphics[width=0.9\linewidth]{img/WAC_Azure_2.png}
		\centering
		\caption{Enter the product key}
		\label{fig:WACAzure2}
	\end{subfigure}
	\begin{subfigure}{0.5\textwidth}
		\captionsetup{width=0.8\linewidth}
		\includegraphics[width=0.9\linewidth]{img/WAC_Azure_3.png} 
		\centering
		\caption{Select a version of choice}
		\label{fig:WACAzure3}
	\end{subfigure}
\end{figure}
\begin{figure}[h]\ContinuedFloat
	\begin{subfigure}{0.5\textwidth}
		\captionsetup{width=0.8\linewidth}
		\includegraphics[width=0.9\linewidth]{img/WAC_Azure_4.png}
		\centering
		\caption{Accept the licence terms}
		\label{fig:WACAzure4}
	\end{subfigure}
	\begin{subfigure}{0.5\textwidth}
		\captionsetup{width=0.8\linewidth}
		\includegraphics[width=0.9\linewidth]{img/WAC_Azure_5.png} 
		\centering
		\caption{Select "Keep personal files and apps" to perform an in-place upgrade}
		\label{fig:WACAzure5}
	\end{subfigure}
\end{figure}
\begin{figure}[h]\ContinuedFloat
	\begin{subfigure}{\textwidth}
		\captionsetup{width=0.8\linewidth}
		\includegraphics[width=0.9\linewidth]{img/WAC_Azure_6.png}
		\centering
		\caption{Review the settings and start the installation}
		\label{fig:WACAzure6}
	\end{subfigure}
	\caption{Windows Admin Center Azure Integration}
	\label{fig:WACAzure}
\end{figure}
%%---------- Creation of container environment --------------------------------------------------
\chapter{Creation of container environment}
\label{Containers_Azure}
\section{Creation of Microsoft Azure \acrshort{vm}}
\begin{figure}[h]
	\begin{subfigure}{\textwidth}
		\captionsetup{width=0.9\linewidth}
		\includegraphics[width=\linewidth]{img/Container_VM_0.png} 
		\centering
		\caption{Add a new \acrshort{vm} through the Azure Portal}
		\label{fig:Container_VM_0}
	\end{subfigure}
\end{figure}
\begin{figure}[h]\ContinuedFloat
	\begin{subfigure}{\textwidth}
		\captionsetup{width=0.9\linewidth}
		\includegraphics[width=\linewidth]{img/Container_VM_1.png}
		\centering
		\caption{Fill in the required parameters and select the correct image}
		\label{fig:Container_VM_1}
	\end{subfigure}
\end{figure}
\begin{figure}[h]\ContinuedFloat
	\begin{subfigure}{\textwidth}
		\captionsetup{width=0.9\linewidth}
		\includegraphics[width=\linewidth]{img/Container_VM_2.png}
		\centering
		\caption{Review the parameters and create the \acrshort{vm}}
		\label{fig:Container_VM_2}
	\end{subfigure}
	\caption{Creation of Microsoft Azure \acrshort{vm}}
	\label{fig:Container_VM}
\end{figure}
\clearpage
\section{Installation of Docker EE}
The following commands have to be run through Powershell on the created \acrshort{vm} to install Docker EE.
\begin{verbatim}
\begin{verbatim}
Install-Module -Name DockerMsftProvider -Repository PSGallery -Force
Install-Package -Name docker -ProviderName DockerMsftProvider
Restart-Computer -Force
\end{verbatim}
\section{Installation of Git}
The following commands have to be run through Powershell on the created \acrshort{vm} to install Git. Git is necessary to clone the files needed to run the benchmarks.
\begin{verbatim}
	Set-ExecutionPolicy Bypass -Scope Process -Force; iex ((New-Object System.Net.WebClient).DownloadString('https://chocolatey.org/install.ps1'))
	choco install git -params '"/GitAndUnixToolsOnPath"'
\end{verbatim}
%%---------- Benchmark data --------------------------------------------------
\chapter{Benchmark Data}
\label{benchmarkdata}
\section{Benchmark Data}
\begin{table}[h]
	\centering
	\begin{tabular}{ll|l|l|l|l|l}
		Server      & Version & Method & N  & Mean    & Error  & StdDev \\ \hline
		Nano Server & 1709    & SHA256 & 10 & 769.900 & 11.909 & 9.945  \\
		Nano Server & 1709    & MD5    & 10 & 516.500 & 8.543  & 7.991  \\
		Nano Server & 1809    & SHA256 & 10 & 781.800 & 13.820 & 12.930 \\
		Nano Server & 1809    & MD5    & 10 & 513.700 & 10.950 & 11.240 \\
		Server Core & 1709    & SHA256 & 10 & 813.900 & 16.370 & 23.478 \\
		Server Core & 1709    & MD5    & 10 & 502.300 & 9.554  & 8.937  \\
		Server Core & 1809    & SHA256 & 10 & 777.900 & 13.352 & 18.275 \\
		Server Core & 1809    & MD5    & 10 & 516.500 & 9.911  & 9.734  \\
		Windows     & 1809    & SHA256 & 10 & 854.200 & 16.730 & 20.543 \\
		Windows     & 1809    & MD5    & 10 & 571.100 & 11.110 & 9.844  \\
		Nano Server & 1709    & SHA256 & 20 & 796.300 & 12.101 & 11.320 \\
		Nano Server & 1709    & MD5    & 20 & 517.400 & 10.550 & 13.465 \\
		Nano Server & 1809    & SHA256 & 20 & 797.800 & 16.060 & 18.490 \\
		Nano Server & 1809    & MD5    & 20 & 531.000 & 10.450 & 14.650 \\
		Server Core & 1709    & SHA256 & 20 & 772.400 & 14.251 & 11.901 \\
		Server Core & 1709    & MD5    & 20 & 503.400 & 9.842  & 12.798 \\
		Server Core & 1809    & SHA256 & 20 & 777.200 & 13.499 & 11.272 \\
		Server Core & 1809    & MD5    & 20 & 508.400 & 10.182 & 10.895 \\
		Windows     & 1809    & SHA256 & 20 & 831.700 & 13.110 & 11.620 \\
		Windows     & 1809    & MD5    & 20 & 568.700 & 15.020 & 17.882
	\end{tabular}
	\label{BenchmarkData}
	\caption{Benchmarking Results}
\end{table}
\section{Graphs}
\begin{figure}[h]
	\begin{subfigure}{\textwidth}
	\captionsetup{width=0.8\linewidth}
	\includegraphics[width=0.9\linewidth]{img/Methodologie/Containers3.png}
	\centering
	\caption{Windows Base Container Image MD5 benchmark (N=10)}
	\end{subfigure}
	\begin{subfigure}{\textwidth}
	\captionsetup{width=0.8\linewidth}
	\includegraphics[width=0.9\linewidth]{img/Methodologie/Containers1.png}
	\centering
	\caption{Windows Base Container Image MD5 benchmark (N=20)}
	\end{subfigure}
	\label{fig:MD5Benchmark}
	\caption[MD5 benchmark]{Windows Base Container Image MD5 benchmark}
\end{figure}
\begin{figure}[h]
	\begin{subfigure}{\textwidth}
	\captionsetup{width=0.8\linewidth}
	\includegraphics[width=0.9\linewidth]{img/Methodologie/Containers4.png}
	\centering
	\caption{Windows Base Container Image SHA-256 benchmark (N=10)}
	\end{subfigure}
	\begin{subfigure}{\textwidth}
	\captionsetup{width=0.8\linewidth}
	\includegraphics[width=0.9\linewidth]{img/Methodologie/Containers2.png}
	\centering
	\caption{Windows Base Container Image SHA-256 benchmark (N=20)}
	\end{subfigure}
	\label{fig:SHABenchmark}
	\caption[SHA-256 benchmark]{Windows Base Container Image SHA-256 benchmark}
\end{figure}
%%---------- Migration --------------------------------------------------
\chapter{Migration}
\label{Migration}
\section{\acrfull{vm} creation}
\begin{figure}[!htb]
	\begin{subfigure}{0.5\textwidth}
		\captionsetup{width=0.8\linewidth}
		\includegraphics[width=0.9\linewidth]{img/Methodologie/Migration0.png}
		\centering
		\caption{Specify name and location}
	\end{subfigure}
	\begin{subfigure}{0.5\textwidth}
		\captionsetup{width=0.8\linewidth}
		\includegraphics[width=0.9\linewidth]{img/Methodologie/Migration1.png} 
		\centering
		\caption{Configure the network connection}
	\end{subfigure}
\end{figure}
\begin{figure}[!htb]\ContinuedFloat
	\begin{subfigure}{0.5\textwidth}
		\captionsetup{width=0.8\linewidth}
		\includegraphics[width=0.9\linewidth]{img/Methodologie/Migration2.png}
		\centering
		\caption{Install an \acrshort{os} from a bootable image}
	\end{subfigure}
	\begin{subfigure}{0.5\textwidth}
		\captionsetup{width=0.8\linewidth}
		\includegraphics[width=0.9\linewidth]{img/Methodologie/Migration3.png} 
		\centering	
		\caption{Verify the \acrshort{vm}}
	\end{subfigure}
	\caption[\acrshort{vm} creation]{Creation of the \acrshort{vm} in Hyper-V}
	\label{fig:VMCreation}
\end{figure}
\section{Windows Server installation}
\begin{figure}[!htb]
	\begin{subfigure}{0.5\textwidth}
		\captionsetup{width=0.8\linewidth}
		\includegraphics[width=0.9\linewidth]{img/Methodologie/Migration4.png}
		\centering
		\caption{Start Windows Server installation}
	\end{subfigure}
	\begin{subfigure}{0.5\textwidth}
		\captionsetup{width=0.8\linewidth}
		\includegraphics[width=0.9\linewidth]{img/Methodologie/Migration5.png} 
		\centering
		\caption{Provide the product key}
	\end{subfigure}
\end{figure}
\begin{figure}[!htb]\ContinuedFloat
	\begin{subfigure}{0.5\textwidth}
		\captionsetup{width=0.8\linewidth}
		\includegraphics[width=0.9\linewidth]{img/Methodologie/Migration6.png}
		\centering
		\caption{Select Windows Server 2019 Datacenter (Desktop Experience)}
	\end{subfigure}
	\begin{subfigure}{0.5\textwidth}
		\captionsetup{width=0.8\linewidth}
		\includegraphics[width=0.9\linewidth]{img/Methodologie/Migration7.png} 
		\centering
		\caption{Agree with the notices and license terms}
	\end{subfigure}
\end{figure}
\begin{figure}[!htb]\ContinuedFloat
	\begin{subfigure}{0.5\textwidth}
		\captionsetup{width=0.8\linewidth}
		\includegraphics[width=0.9\linewidth]{img/Methodologie/Migration8.png}
		\centering
		\caption{Install Windows only}
	\end{subfigure}
	\begin{subfigure}{0.5\textwidth}
		\captionsetup{width=0.8\linewidth}
		\includegraphics[width=0.9\linewidth]{img/Methodologie/Migration9.png} 
		\centering	
		\caption{Customize the final settings}
	\end{subfigure}
	\caption[Installing Windows Server]{Installing Windows Server 2019 Datacenter Edition on the \acrshort{vm}}
	\label{fig:WSInstallation}
\end{figure}
\clearpage
\section{Joining the domain}
\begin{figure}[!htb]
	\begin{subfigure}{0.5\textwidth}
		\captionsetup{width=0.8\linewidth}
		\includegraphics[width=0.9\linewidth]{img/Methodologie/Migration10.png}
		\centering
		\caption{Click the current Workgroup}
	\end{subfigure}
	\begin{subfigure}{0.5\textwidth}
		\captionsetup{width=0.8\linewidth}
		\includegraphics[width=0.9\linewidth]{img/Methodologie/Migration11.png} 
		\centering
		\caption{Select and enter the required domain}
	\end{subfigure}
\end{figure}
\begin{figure}[!htb]\ContinuedFloat
	\begin{subfigure}{0.5\textwidth}
		\captionsetup{width=0.8\linewidth}
		\includegraphics[width=0.9\linewidth]{img/Methodologie/Migration12.png}
		\centering
		\caption{Provide domain credentials}
	\end{subfigure}
	\begin{subfigure}{0.5\textwidth}
		\captionsetup{width=0.8\linewidth}
		\includegraphics[width=0.9\linewidth]{img/Methodologie/Migration13.png} 
		\centering	
		\caption{Verify the domain changes}
	\end{subfigure}
	\caption[Joining the domain]{Joining the corp.contoso.com domain}
	\label{fig:Domain}
\end{figure}
\section{Promoting the server to \acrshort{dc}}
\begin{figure}[!htb]
	\begin{subfigure}{0.5\textwidth}
		\captionsetup{width=0.8\linewidth}
		\includegraphics[width=0.9\linewidth]{img/Methodologie/Migration15.png}
		\centering
		\caption{Start the roles and features wizard}
	\end{subfigure}
	\begin{subfigure}{0.5\textwidth}
		\captionsetup{width=0.8\linewidth}
		\includegraphics[width=0.9\linewidth]{img/Methodologie/Migration16.png} 
		\centering
		\caption{Select the current server}
	\end{subfigure}
\end{figure}
\begin{figure}[!htb]\ContinuedFloat
	\begin{subfigure}{0.5\textwidth}
		\captionsetup{width=0.8\linewidth}
		\includegraphics[width=0.9\linewidth]{img/Methodologie/Migration17.png}
		\centering
		\caption{Select the required server roles}
	\end{subfigure}
	\begin{subfigure}{0.5\textwidth}
		\captionsetup{width=0.8\linewidth}
		\includegraphics[width=0.9\linewidth]{img/Methodologie/Migration18.png} 
		\centering
		\caption{Promote the server to \acrshort{dc}}
	\end{subfigure}
\end{figure}
\begin{figure}[!htb]\ContinuedFloat
	\begin{subfigure}{0.5\textwidth}
		\captionsetup{width=0.8\linewidth}
		\includegraphics[width=0.9\linewidth]{img/Methodologie/Migration19.png}
		\centering
		\caption{Add a new \acrshort{dc} to an existing domain}
	\end{subfigure}
	\begin{subfigure}{0.5\textwidth}
		\captionsetup{width=0.8\linewidth}
		\includegraphics[width=0.9\linewidth]{img/Methodologie/Migration20.png} 
		\centering
		\caption{Verify DSRM password}
	\end{subfigure}
\end{figure}
\begin{figure}[!htb]\ContinuedFloat
	\begin{subfigure}{0.5\textwidth}
		\captionsetup{width=0.8\linewidth}
		\includegraphics[width=0.9\linewidth]{img/Methodologie/Migration21.png}
		\centering
		\caption{Replicate from the previous \acrshort{dc}}
	\end{subfigure}
	\begin{subfigure}{0.5\textwidth}
		\captionsetup{width=0.8\linewidth}
		\includegraphics[width=0.9\linewidth]{img/Methodologie/Migration22.png} 
		\centering	
		\caption{Verifying and continue the promotion}
	\end{subfigure}
	\caption[\acrshort{dc} promotion]{Promoting the server to \acrshort{dc}}
	\label{fig:Promotion}
\end{figure}
\clearpage
\section{Migrating FSMO roles}
\begin{figure}[!htb]
		\captionsetup{width=0.8\linewidth}
		\includegraphics[width=0.9\linewidth]{img/Methodologie/Migration23.png} 
		\centering	
		\caption[FSMO migration]{Migrate the FSMO roles to the new \acrshort{dc}}
		\label{fig:FSMO}
\end{figure}
\clearpage
\section{Configuring DNS and DHCP}
\begin{figure}[!htb]
	\begin{subfigure}{0.5\textwidth}
		\captionsetup{width=0.8\linewidth}
		\includegraphics[width=0.9\linewidth]{img/Methodologie/Migration24.png}
		\centering
		\caption{DNS is automatically replicated}
	\end{subfigure}
	\begin{subfigure}{0.5\textwidth}
		\captionsetup{width=0.8\linewidth}
		\includegraphics[width=0.9\linewidth]{img/Methodologie/Migration25.png} 
		\centering
		\caption{Export the DHCP configuration}
	\end{subfigure}
\end{figure}
\begin{figure}[!htb]\ContinuedFloat
	\begin{subfigure}{0.5\textwidth}
		\captionsetup{width=0.8\linewidth}
		\includegraphics[width=0.9\linewidth]{img/Methodologie/Migration26.png}
		\centering
		\caption{ Install the required roles on the new \acrshort{dc}}
	\end{subfigure}
	\begin{subfigure}{0.5\textwidth}
		\captionsetup{width=0.8\linewidth}
		\includegraphics[width=0.9\linewidth]{img/Methodologie/Migration27.png} 
		\centering	
		\caption{Import the DHCP configuration}
	\end{subfigure}
	\caption[Configuring DHCP \& DNS]{Configuring DNS and DHCP on the new \acrshort{dc}}
	\label{fig:DNSDHCP}
\end{figure}
\clearpage
\section{Decommissioning the old \acrshort{dc}}
\begin{figure}[!htb]
	\begin{subfigure}{0.5\textwidth}
		\captionsetup{width=0.8\linewidth}
		\includegraphics[width=0.9\linewidth]{img/Methodologie/Migration29.png}
		\centering
		\caption{Remove roles and features}
	\end{subfigure}
	\begin{subfigure}{0.5\textwidth}
		\captionsetup{width=0.8\linewidth}
		\includegraphics[width=0.9\linewidth]{img/Methodologie/Migration30.png} 
		\centering
		\caption{Select the current \acrshort{dc}}
	\end{subfigure}
\end{figure}
\begin{figure}[!htb]\ContinuedFloat
	\begin{subfigure}{0.5\textwidth}
		\captionsetup{width=0.8\linewidth}
		\includegraphics[width=0.9\linewidth]{img/Methodologie/Migration31.png}
		\centering
		\caption{Demote the old \acrshort{dc}}
	\end{subfigure}
	\begin{subfigure}{0.5\textwidth}
		\captionsetup{width=0.8\linewidth}
		\includegraphics[width=0.9\linewidth]{img/Methodologie/Migration32.png} 
		\centering
		\caption{Verify the removal of the \acrshort{dc}}
	\end{subfigure}
\end{figure}
\begin{figure}[!htb]\ContinuedFloat
	\begin{subfigure}{0.5\textwidth}
		\captionsetup{width=0.8\linewidth}
		\includegraphics[width=0.9\linewidth]{img/Methodologie/Migration33.png}
		\centering
		\caption{Uncheck removal of the DNS Delegation}
	\end{subfigure}
	\begin{subfigure}{0.5\textwidth}
		\captionsetup{width=0.8\linewidth}
		\includegraphics[width=0.9\linewidth]{img/Methodologie/Migration34.png} 
		\centering	
		\caption{Demote the \acrshort{dc}}
	\end{subfigure}
	\caption[Decommissioning the \acrshort{dc}]{Decommissioning the old \acrshort{dc}}
	\label{fig:Decomissioning}
\end{figure}
%%---------- Referentielijst --------------------------------------------------
\printbibliography[heading=bibintoc]
\end{document}