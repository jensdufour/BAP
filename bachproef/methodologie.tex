% !TeX spellcheck = en_GB
%%=============================================================================
%% Methodologie
%%=============================================================================

\chapter{\IfLanguageName{dutch}{Methodologie}{Methodology}}
\label{ch:methodologie}
% Tip: Begin elk hoofdstuk met een paragraaf inleiding die beschrijft hoe dit hoofdstuk past binnen het geheel van de bachelorproef. Geef in het bijzonder aan wat de link is met het vorige en volgende hoofdstuk.

% TODO: Hoe ben je te werk gegaan? Verdeel je onderzoek in grote fasen, en licht in elke fase toe welke stappen je gevolgd hebt. Verantwoord waarom je op deze manier te werk gegaan bent. Je moet kunnen aantonen dat je de best mogelijke manier toegepast hebt om een antwoord te vinden op de onderzoeksvraag:

% What are the advantages and disadvantages of a migration from Windows Server 2016 to 2019 in a business environment?
%% What are the differences between the standard, ServerCore and Nano version of Windows Server 2019? Deze zijn virtualizatiecontainers!
%% Can we migrate SAP from an existing Windows Server 2016 to 2019 in a business environment?
%% How can the new features of Windows Server 2019 be leveraged in the migrated infrastructure? 

In this chapter the upgrade from Windows Server 2016 to Windows Server 2019 will be researched. This upgrade will first be performed using the in-place upgrade method after which the process will be repeated using a migration. After this the migration of a typical SAP environment, as described by Delaware, from Windows Server 2016 to Windows Server 2019 will be examined. To continue the different versions of the container images will be analysed and how these can lower virtual machine overhead, and improve virtualization efficiency. Finally, their will be a discussion about how the new features that come with Windows Server 2019 can be leveraged in the migrated infrastructure. But first the used infrastructure will be discussed.

\section{Technical specifications of the proof of concept setup}
The proof of concept was made on a bare-metal server running Windows Server 2016. The proof of concept environment was than virtualized using Hyper-V. 
\begin{itemize}
	\item CPU: Intel Xeon E5620
	\item RAM: 96 GB 
	\item HDD: 500 GB
	\item OS Version: Windows Server 2016
	\item Hyper-V role installed
	\item Administrative rights on the device
\end{itemize}
The proof of concept environment was based on the Modern Desktop Deployment and Management Lab Kit provided by Microsoft. \autocite{Gallagher2018}
This to make replication of the proof of concept simple and efficient.
\\
The Modern Desktop Deployment and Management Lab Kit consists of the following components:
\\

\begin{table}[ht]
	\centering
	\begin{adjustbox}{width=1\textwidth}
	\begin{tabular}{l|l}
		Server Name  & Roles \& Products                                                  	 \\ 
		\hline
		HYD-DC1      & Active Directory Domain Controller, DNS, DHCP, Certificate Services 	   \\
		HYD-MDT1     & Microsoft Deployment Toolkit                                        		\\
		& Windows 10 1809 ADK                                                 					 \\
		& Windows Deployment Services                                         					  \\
		HYD-CM1      & System Center Configuration Manager 1806                            		   \\
		& Windows Deployment Services                                         						\\
		& Microsoft Deployment Toolkit                                         						 \\
		& Windows 10 1809 ADK                                                 			 			  \\
		& Windows Software Update Services                                        		  			   \\
		& Microsoft SQL Server 2014                                           						    \\
		HYD-APP1     & Microsoft BitLocker Administration and Monitoring                   				 \\
		& Microsoft SQL Server 2014                                          							  \\
		HYD-GW1      & Remote Access for Internet Connectivity                           				   \\
		HYD-INET1    & Simulated Internet                                                			 	    \\
		HYD-VPN1     & Remote Access for VPN                                             				     \\
		HYD-CLIENT1  & Windows 10 1809 Domain Joined                                  					      \\
		& Office 365 ProPlus Build 16.0.11121.20000                         								   \\
		HYD-CLIENT2  & Windows 10 1809 Domain Joined                                     					    \\
		& Office 365 ProPlus Build 16.0.11121.20000                           									 \\
		HYD-CLIENT3  & Windows 10 1809 Workgroup                                         						  \\
		HYD-CLIENT4  & Windows 10 1809 Workgroup                                          						   \\
		HYD-CLIENT5 & Bare metal (no installations)                                      						    \\
		HYD-CLIENT6 & Bare metal (no installations)                                       							 \\
		HYD-CLIENT7  & Windows 7 Domain Joined                                            
	\end{tabular}
	\end{adjustbox}
	\caption[Lab Kit Components]{Modern Desktop Deployment and Management Lab Kit Components}
	\scriptsize	
	Adapted from \cite{MicrosoftCorporation2019}
	%TODO Bronvermelding correct?
	\label{tab:MDDMLK2016}
\end{table}

	
Only HYD-CLIENT1 will be kept. Connection to the other \acrshort{vm}s can be made using the following credentials:
\begin{table}[ht]
	\centering
	\begin{adjustbox}{width=1\textwidth}
	\begin{tabular}{l|lll}
		User                 & Access Type              & User Name                    & Password \\
		\hline
		Local Administrator  & Administrative           & Administrator                & P@ssw0rd \\
		Domain Administrator & Enterprise Administrator & CORP\textbackslash{}LabAdmin & P@ssw0rd
	\end{tabular}
	\end{adjustbox}
	\caption[Lab Kit Credentials]{Modern Desktop Deployment and Management Lab Kit Credentials}
	\scriptsize	
	Adapted from \cite{MicrosoftCorporation2019}
	%TODO Bronvermelding correct?
	\label{tab:MDDMLK2016}
\end{table}

Additionally, Windows Admin Center will be installed manually to HYD-DC1. 


\section{Upgrading the \acrshort{os}}
\subsection{In-place upgrade}
\subsection{Migration}
\section{SAP migration}
\section{Windows, Windows Server Core and Nano Server}
